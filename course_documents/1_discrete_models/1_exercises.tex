\documentclass[14pt]{amsart}
%\usepackage{amsmath}
\usepackage[utf8]{inputenc}
\usepackage{graphicx}
\usepackage{xcolor}
\usepackage[legalpaper, margin=1.0in]{geometry}
\usepackage{fancyvrb}
\usepackage{cprotect}
\usepackage{url}
\usepackage{etoolbox}
\usepackage{hyperref}
\usepackage[skins,theorems]{tcolorbox}
\usepackage{enumitem}
\newcommand*{\vertbar}{\rule[-1ex]{0.5pt}{2.5ex}}
\newcommand*{\horzbar}{\rule[.5ex]{2.5ex}{0.5pt}}
\setlist[itemize]{align=parleft,left=1pt..1em}

\usepackage{imakeidx}
\makeindex


\usepackage{cmbright}
\usepackage[OT1]{fontenc}


% fonts
%\input{ArtNouvc.fd}
%\newcommand*\initfamily{\usefont{U}{ArtNouvc}{xl}{n}}
%\usepackage[T1]{fontenc}
%\usepackage{tgbonum} % change font

%\usepackage[light,condensed,math]{iwona}
%\usepackage[T1]{fontenc}

% from https://www.pinterest.co.uk/pin/88664686405122253/
\definecolor{C1}{RGB}{141, 125, 158} %purple
\definecolor{C2}{RGB}{163,156,147} % yellow
\definecolor{C3}{RGB}{99,151,153} % green
\definecolor{C4}{RGB}{195,106,99} % red
\definecolor{C5}{RGB}{124, 132, 128}
\definecolor{C6}{RGB}{67, 69, 75}


\definecolor{Gr}{HTML}{377D71}
\definecolor{Pu}{HTML}{A459D1}
\definecolor{Bl}{HTML}{4D455D}
\definecolor{Te}{HTML}{C1ECE4}
\definecolor{Or}{HTML}{EF6262}
\definecolor{Am}{HTML}{F3AA60}
\definecolor{Co}{HTML}{3C486B}
\definecolor{Wh}{HTML}{FEFBF6}
\definecolor{Ye}{HTML}{FFE196}
\definecolor{Re}{HTML}{E96479}
\definecolor{Pi}{HTML}{FFD0D0}
\definecolor{Rp}{HTML}{FF9EAA}
\definecolor{Wg}{HTML}{EEF3D2}


\hypersetup{
    colorlinks=false,
    linkcolor=red,
    filecolor=red,      
    urlcolor=red,
    pdftitle={a},
    pdfpagemode=FullScreen,
    }
    
    
 % TCOLORBOX STUFF ----------------------------------------------------------------------------
 

 %  ----------------------------------------------------------------------------
    
\patchcmd{\section}{\normalfont}{\color{C5}}{}{}
\patchcmd{\subsection}{\normalfont}{\color{C5}}{}{}


\newcommand{\dfn}[1]{{\bf  \color{blue}{#1}}}


\renewcommand{\FancyVerbFormatLine}[1]{\color{gray}{>\,\,#1}}
    
\newtheorem{thm}{Theorem}


\newtheoremstyle{exercise}
{}                % Space above
{}                % Space below
{\color{black}}        % Theorem body font % (default is "\upshape")
{}                % Indent amount
{\bfseries}       % Theorem head font % (default is \mdseries)
{:}               % Punctuation after theorem head % default: no punctuation
{ }               % Space after theorem head
{}                % Theorem head spec
\theoremstyle{exercise}
\newtheorem{exercise}{Exercise}


\newtheoremstyle{example}
{}                % Space above
{}                % Space below
{\color{red}\slshape}        % Theorem body font % (default is "\upshape")
{}                % Indent amount
{\bfseries}       % Theorem head font % (default is \mdseries)
{.}               % Punctuation after theorem head % default: no punctuation
{ }               % Space after theorem head
{}                % Theorem head spec
\theoremstyle{example}
\newtheorem{example}{Example}

\newtheoremstyle{solution}
{}                % Space above
{}                % Space below
{\color{C5}}        % Theorem body font % (default is "\upshape")
{}                % Indent amount
{\bfseries }       % Theorem head font % (default is \mdseries)
{:}               % Punctuation after theorem head % default: no punctuation
{\newline}               % Space after theorem head
{}                % Theorem head spec
\theoremstyle{solution}
\newtheorem*{solution}{Solution}


\newcommand{\mcS}{\mathcal S}
\newcommand{\mcR}{\mathcal R}
\newcommand{\mcC}{\mathcal C}
\newcommand{\bS}{{\boldsymbol S}}
\newcommand{\bR}{{\boldsymbol R}}
\newcommand{\bC}{{\boldsymbol C}}
\newcommand{\ba}{{\boldsymbol a}}
\newcommand{\bb}{{\boldsymbol b}}
\newcommand{\bs}{{\boldsymbol s}}
\newcommand{\bff}{{\boldsymbol f}}
\newcommand{\br}{{\boldsymbol r}}
\newcommand{\bx}{{\boldsymbol x}}
\newcommand{\bt}{{\boldsymbol t}}
\newcommand{\bv}{{\boldsymbol v}}
\newcommand{\bu}{{\boldsymbol u}}
\newcommand{\bw}{{\boldsymbol w}}
\newcommand{\bc}{{\boldsymbol c}}
\newcommand{\be}{{\boldsymbol e}}
\newcommand{\bq}{{\boldsymbol q}}
\newcommand{\bphi}{{\boldsymbol \phi}}
\newcommand{\brho}{{\boldsymbol \rho}}
\newcommand{\btau}{{\boldsymbol \tau}}
\newcommand{\reals}{\mathbb R}
\newcommand{\ints}{\mathbb N}
\newcommand{\E}{\mathbb E}
\newcommand{\Prob}{\mathbb P}


%%%%%%%%%%%%%%%




%--------------------------------------------------------------------------------------------------------------------------------
\begin{document}


\title{Exercise set 1}

\maketitle






\section{Exercises}


%--------------------------------------------------------------------------------------------------------------------------------



%--------------------------------------------------------------------------------------------------------------------------------
\begin{exercise}[Working with probability distributions and modeling]
The first two problems are inspired by those in section 2 of \cite{tabak}.   You should look there for more practice. 
\begin{enumerate}[label=(\alph*)]
\item Suppose that 
\begin{equation*}
Y \sim {\rm Bernoulli}(q)
\end{equation*}
and let $Z = 1/(1+Y) + Y$. What is the sample space of $Z$ and what is the probability function of $Z$? You can either write the probability distribution as a piecewise function as I did for the Bernoulli distribution in class, or specify each value e.g. $P(Z=z) = \cdots$.
\item Suppose a coin is flipped. If the coin is heads, we write down $0$. If the coin is tails, we roll a dice and write down the number. What is the sample space and the probability distribution for $Y$, the number that we write down. 
\item For the previous problem, conditioned on the dice rolling a $4$, what is the probability we write down $0$? Conditioned on the coin being tails, what is the probability the dice rolls a $3$?
\item Consider the geometric distribution discussed in lecture. What are 3 examples of variables in the real world for which this might be a good model and what are some limitations of these models. 
\end{enumerate}
\end{exercise}




%--------------------------------------------------------------------------------------------------------------------------------
 \begin{exercise}[Working with nested for loops]
 
 

Consider the following code:

\begin{Verbatim}
for i in range(5):
  for j in range(i+1):
    print(i,end='')
  print("")
\end{Verbatim}
prints out
\begin{Verbatim}
0
11
222
3333
44444
\end{Verbatim}
Without using ChatGPT, modify this code to print
\begin{Verbatim}
0
01
012
0123
01234
012345
\end{Verbatim}
\end{exercise}


%--------------------------------------------------------------------------------------------------------------------------------
 \begin{exercise}[Working with more complex data -- {\bf ChatGPT}]
 Using ChatGPT, write python code to plot a map of the world with Hanover indicated by a red star. 
 Then examine this code and answer the following questions: 
 \begin{itemize}
 \item How was the data loaded and what variable was it stored in?  Where is the information about the geometric shape of each country stored? Can you print out some of this information? Is there other information that is not used in the plot? 
 \item Where is the information about the location of Hanover, NH stored? 
 \item {\bf Without using ChatGPT}, could plot another point at Salt Lake City, UT with a green dot? (you can look up the coordinates). 
 \end{itemize}
 
 \end{exercise}

 
%--------------------------------------------------------------------------------------------------------------------------------
% \begin{exercise}[Working with housing data] 
% This problem is based on Ch. 2 Ex. 10 in \cite{islp}. Use the following code to load data on housing Boston from \cite{islp}. The data is described \href{https://islp.readthedocs.io/en/latest/datasets/Boston.html}{here}.
% \begin{Verbatim}
%Boston = pd.read_csv("https://raw.githubusercontent.com
%/intro-stat-learning/ISLP/main/ISLP/data/Boston.csv")
%\end{Verbatim}
%\begin{enumerate}[label=(\alph*)]
%\item How many rows are in this data set? How many columns? What do the rows and columns represent?
%%\item  Make pairwise scatterplots of the 
%%\item Are any of the predictors associated with per capita crime rate? If so, explain the relationship.
%\item How many of the suburbs in this data set bound the Charles river?
%\end{enumerate}
% \end{exercise}
 
%--------------------------------------------------------------------------------------------------------------------------------
\begin{exercise}[Washington post data]\label{ex:washpost}
Below I load some data on homocide victims in US from the washington post. Don't worry about how I process it, all you need to work with is the DataFrame ``data`` on the very last line.
%(When I first tried to run this line of code I got an error about the encoding
%a quick google search led to a post which suggested I should add the line 
% "encoding = "ISO-8859-1"""  I have no clue what this means, but it seems to get rid of the problem.)

\begin{Verbatim}
data = pd.read_csv("https://raw.githubusercontent.com/washingtonpost
/data-homicides/master/homicide-data.csv",encoding = "ISO-8859-1")
data["victim_age"] = pd.to_numeric(data["victim_age"],errors="coerce")
\end{Verbatim}
%
%the next hing I noticed is that the ages are not number, but strings
%I once again did a quick google search for "converting numpy columns to numbers" and learned I had to do the following
%\begin{Verbatim}
%\end{Verbatim}

\begin{enumerate}[label=(\alph*)]
\item  For each age $a = 1,\cdots,100$ determine the number of victims $n(a)$ with an age $<a$ and put these values in a list. You can ignore the effects of those entries with missing ages. 
\item Now think for a moment about what you expect a plot of $n(a) vs. a$ to look like, then make a plot of $n(a)$ vs. $a$. Does it look like as expected?
\item Break the data up into white and non-white victims and repeat part (a) for each group. Then, for each group, make the plot from part (a). Comment on what you find. 
\end{enumerate}
%You can break this data up into the entries where the victims is white and non-white with the code
%
%\begin{Verbatim}
%data_white = data[data.victim_race == "White"]
%data_notwhite = data[data.victim_race != "White"]
%\end{Verbatim}

\end{exercise}



%--------------------------------------------------------------------------------------------------------------------------------
\begin{exercise}[Getting a sequence of wins]
Let $J$ denote a random variable representing the number of times a fair coin is flipped before two heads appear in a row. As we saw in class, the following code generates simulations of $J$:
\begin{Verbatim}
def flip_until_two():
  num_heads = 0
  total_flips = 0
  while num_heads <2:
    y = np.random.choice([0,1])
    if y == 0:
      num_heads = 0
    else:
      num_heads = num_heads + 1
    total_flips = total_flips + 1
  return total_flips
\end{Verbatim}

\begin{enumerate}[label=(\alph*)]
\item  Without using ChatGPT, by changing the code above, write a function rolluntil(n) that rolls a dice until we
get $n$ ones in a row. You should change the variable names accordingly. We will call this random variable $R_n$.
%\item Derive a formula for $P(J=k)$. Hint: it is similar to the geometric distribution. 
\item  Make a DataFrame where each column represents a value of $n$ from $1$ to $6$ and each row is a simulation from the model $R_n$. There should be $100$ rows.
\item Make a plot of the maximum and minimum values of $R_n$ as a function of $n$ on the same plot. You might notice one of these increases much faster than the other.  
\end{enumerate}
\end{exercise}


%--------------------------------------------------------------------------------------------------------------------------------
%\begin{exercise}[The binomial distribution]
%Let $X_1,\dots,X_N$ be independent and identically distribution (abbreviated as iid) samples from a Bernoulli distribution with probability of success $q=1$. Consider the sum 
%\begin{equation}
%S_N = \sum_{i=1}^N X_i
%\end{equation}
%The variable $S_N$ is called a binomial random variable -- we can think of it as counting the number of ``successes'' in a sequence of $N$ trials which have a probability of success $p$. 
%The binomial distribution has the PMF
%\begin{equation}\label{eq:pdf-binomial}
%P(S_N = k) = {N \choose k}p^{k}(1-p)^{k-1}
%\end{equation}
%\begin{enumerate}[label=(\alph*)]
%\item What is the chance that $S_N = N$, what about $S_3 = 2$? (Hint: Write out all the outcomes) \label{ex:1a}
%\item Write a python function which generates samples of a Binomial distribution by filling in the code
%\begin{Verbatim}
%def my_binomial(N,q,n_samples):
%  y = np.zeros(n_samples)
%  for i in range(n_samples):
%    # fill in code here
%  return y
%\end{Verbatim}
%\item Using Monte Carlo simulations, test if your answers to \ref{ex:1a} are correct. 
%\item  Using Monte Carlo simulations, make a plot of $P(S_N= k)$ as a function of $k$ for $N=100$ and $N=1000$. Compare you simulations to Equation \ref{eq:pdf-binomial} by plotting both on the same plot. 
%\end{enumerate}
%Useful code can be found \href{https://colab.research.google.com/drive/1Gs-gSsUP1hHVwhrbwvWzLVm1ulcLJKRI#scrollTo=_c4br6SCUtUy}{here}
%%
%\end{exercise}



%
%\begin{exercise}
%\href{https://colab.research.google.com/drive/1Gs-gSsUP1hHVwhrbwvWzLVm1ulcLJKRI#scrollTo=69BY8z2IRXnU&line=6&uniqifier=1}{Verifying an analytical formula with simulations}
%\end{exercise}


%---------------------------------------------------------------------------------------------------------------------------------------
\begin{exercise}[Two gene model]
Consider the variant of the model discussed in class:
\begin{equation*}\label{eq:gene}
\Prob(Y_A,Y_B) = \left\{ \begin{array}{cc}
1/3 & \text{ if }Y_A=0 \text{ and } Y_B = 0\\
1/3 & \text{ if }Y_A=0 \text{ and } Y_B = 1\\
1/6 & \text{ if }Y_A=1 \text{ and } Y_B = 0\\
1/6 & \text{ if }Y_A=1 \text{ and } Y_B = 1\\
\end{array}
 \right.
\end{equation*}
\begin{enumerate}[label=(\alph*)]
\item What are the marginal distributions of $Y_A$ and $Y_B$?
\item Are $Y_A$ and $Y_B$ independent? 
\item Without using ChatGPT, Confirm you answer with simulations. 
\end{enumerate}
%Below I've generated monte carlo simulations from the gene model
%%. Let's pretend you don't know the parameter values (you could figure them out from my code). Supposing you didn't know whether $Y_A$ and $Y_B$ are independent, how would you determine this based on the data?
%\begin{Verbatim}
%yB = np.random.choice([0,1],10000)
%qs =  [yBi*0.3 + (1-yBi)*0.1 for yBi in yB]
%yA = [np.random.choice([0,1],p = [q,1-q]) for q in qs]
%data = np.transpose([yA,yB])
%df = pd.DataFrame(data,columns = ["yA","yB"])
%df
%\end{Verbatim}
%
%\begin{enumerate}[label=(\alph*)]
%\item Based on this code, can you write down a probability model for the variables $Y_A$ and $Y_B$? Are they independent? 
%\item Confirm you answer to the first part of this question using Monte Carlo simulations
%\end{enumerate}

%\href{https://colab.research.google.com/drive/1Gs-gSsUP1hHVwhrbwvWzLVm1ulcLJKRI#scrollTo=HT5mXESCXWYx}{Estimating conditional probability of dice}
\end{exercise}


%---------------------------------------------------------------------------------------------------------------------------------------
\begin{exercise}[Verifying variance formula for Bernoulli variable]
Verify the formula for the variance
\begin{equation*}
{\rm Var}(Y) = q(1-q)
\end{equation*}

Remember, you can do this you can use the fact that pointwise arithmetic between numpy arrays can be performed directly on the ways, e.g.
\begin{Verbatim}
q_range*q_range
\end{Verbatim}
makes a list where every element is the corresponding element of qrange squared. You should experiment to ensure you are using enough samples.
\end{exercise}

%---------------------------------------------------------------------------------------------------------------------------------------
\begin{exercise}[Working with Washington Post Data]

This a continuation of Exercise \ref{ex:washpost}
Consider the quantities
\begin{align*}
&P(age <z)\\
&P(age <z|\text{white})\\
&P(age <z|\text{not white}).
\end{align*}
\begin{enumerate}[label=(\alph*)]
\item Explain who each of these are related to the plot you made in Exercise \ref{ex:washpost}. 
\item Make plots of them and comment of the difference between the plot in Exercise \ref{ex:washpost}. Do you think age and race are independent based on these plots. 
\item Using the data, approximate, 
\begin{equation*}
P(\text{white}| 10<age<60)
\end{equation*}
Hint: One way to do this is to use Bayes' rule 
\end{enumerate}

%\href{https://colab.research.google.com/drive/1Gs-gSsUP1hHVwhrbwvWzLVm1ulcLJKRI#scrollTo=vogWBcGHaZDM&line=3&uniqifier=1}{Working with homocide data}
\end{exercise}


%--------------------------------------------------------------------------------------------------------------------------------
\begin{exercise}[Covid modeling]
Suppose we are interested in modeling how likely we are to contract covid after a night out. Imagine that you interact with $N$ people. Let $Y_i$ represent whether or not the $i$th person you interacted with has covid and $T_i$ represent whether or not you contract covid from the interaction with the $i$th person. 

Our model is as follows:
\begin{align*}
Y_i &\sim {\rm Bernoulli}(1/10)\\
T_i |(Y_i=1) &\sim {\rm Bernoulli}(1/2)\\
T_i|(Y_i=0) &\sim {\rm Bernoulli}(0)
\end{align*}

Answer these questions without ChatGPT. 

\begin{enumerate}[label=(\alph*)]
\item What is the distribution $T_i$ NOT conditioned on $Y_i$.  That is, what is the marginal distribution of $T_i$?
\item Fill in the question marks in the following function so that it simulates whether or not you got covid from the night out; that is, so it returns $1$ if you got covid and $0$ if you didn't.
\begin{Verbatim}
def sim_covid(n):
  got_covid = 0
  for k in range(n):
      got_covid_interaction = ????
      if got_covid_interaction ==1:
        got_covid =1
  return got_covid
\end{Verbatim}
\item  Confirm with Monte Carlo simulations simulations that the probability of getting covid form the entire night out is
\begin{equation}
P(\text{get covid}) = 1-\left(1-\frac{1}{20}\right)^n
\end{equation}
You should make a plot of this probability vs. $n$, similar to what we did for the Bernoulli distribution in the class notebook.
\end{enumerate}
%\href{https://colab.research.google.com/drive/1Gs-gSsUP1hHVwhrbwvWzLVm1ulcLJKRI#scrollTo=TnLORrmyBn6q&line=22&uniqifier=1}{Simulating covid}
\end{exercise}

\bibliographystyle{unsrt}
\bibliography{./../refs.bib}

\end{document}


\begin{exercise}{Simulating Binomial Random variables}
\begin{enumerate}[label=(\alph*)]
\item  Write a function that generates samples from a Binomial distribution by filling in the code below. 
\begin{Verbatim}
def my_binomial(N,q,n_samples):
  y = np.zeros(n_samples)
  for i in range(n_samples):
    # fill in code here
  return y
\end{Verbatim}
\item By running Monte carlo simulations and plotting histograms, compare the samples generated by this code to the sampled generated by the numpy function ``np.random.binomial`` (they should be the same). Along with the histograms, plot the probability distribution. You can use the following function 
\begin{Verbatim}
import scipy.special
def binomial_prob(y,N,q):
  return scipy.special.binom(N, y)*q**y*(1-q)**(N-y)
\end{Verbatim}
You should play around with $N$ and $q$, but only need to make one plot. 
The following code provides an example of how to compare two histograms:

\begin{Verbatim}
fig,ax = plt.subplots(figsize=(4,2))
# y1 and y2 are the two data sets you want to compare
ax.hist([y1,y2],label=["my function","numpy"],density=True)
ax.legend()
\end{Verbatim}
\end{enumerate}



 \bibliographystyle{unsrt}
\bibliography{./../refs.bib}


\end{exercise}