\documentclass[14pt]{amsart}
%\usepackage{amsmath}
\usepackage[utf8]{inputenc}
\usepackage{graphicx}
\usepackage{xcolor}
\usepackage[legalpaper, margin=1.0in]{geometry}
\usepackage{fancyvrb}
\usepackage{cprotect}
\usepackage{url}
\usepackage{etoolbox}
\usepackage{hyperref}
\usepackage[skins,theorems]{tcolorbox}
\usepackage{enumitem}
\newcommand*{\vertbar}{\rule[-1ex]{0.5pt}{2.5ex}}
\newcommand*{\horzbar}{\rule[.5ex]{2.5ex}{0.5pt}}
\setlist[itemize]{align=parleft,left=1pt..1em}

\usepackage{imakeidx}
\makeindex


\usepackage{cmbright}
\usepackage[OT1]{fontenc}


% fonts
%\input{ArtNouvc.fd}
%\newcommand*\initfamily{\usefont{U}{ArtNouvc}{xl}{n}}
%\usepackage[T1]{fontenc}
%\usepackage{tgbonum} % change font

%\usepackage[light,condensed,math]{iwona}
%\usepackage[T1]{fontenc}

% from https://www.pinterest.co.uk/pin/88664686405122253/
\definecolor{C1}{RGB}{141, 125, 158} %purple
\definecolor{C2}{RGB}{163,156,147} % yellow
\definecolor{C3}{RGB}{99,151,153} % green
\definecolor{C4}{RGB}{195,106,99} % red
\definecolor{C5}{RGB}{124, 132, 128}
\definecolor{C6}{RGB}{67, 69, 75}


\definecolor{Gr}{HTML}{377D71}
\definecolor{Pu}{HTML}{A459D1}
\definecolor{Bl}{HTML}{4D455D}
\definecolor{Te}{HTML}{C1ECE4}
\definecolor{Or}{HTML}{EF6262}
\definecolor{Am}{HTML}{F3AA60}
\definecolor{Co}{HTML}{3C486B}
\definecolor{Wh}{HTML}{FEFBF6}
\definecolor{Ye}{HTML}{FFE196}
\definecolor{Re}{HTML}{E96479}
\definecolor{Pi}{HTML}{FFD0D0}
\definecolor{Rp}{HTML}{FF9EAA}
\definecolor{Wg}{HTML}{EEF3D2}


\hypersetup{
    colorlinks=false,
    linkcolor=red,
    filecolor=red,      
    urlcolor=red,
    pdftitle={a},
    pdfpagemode=FullScreen,
    }
    
    
 % TCOLORBOX STUFF ----------------------------------------------------------------------------
 

 %  ----------------------------------------------------------------------------
    
\patchcmd{\section}{\normalfont}{\color{C5}}{}{}
\patchcmd{\subsection}{\normalfont}{\color{C5}}{}{}


\newcommand{\dfn}[1]{{\bf  \color{blue}{#1}}}


\renewcommand{\FancyVerbFormatLine}[1]{\color{gray}{>\,\,#1}}
    
\newtheorem{thm}{Theorem}


\newtheoremstyle{exercise}
{}                % Space above
{}                % Space below
{\color{black}}        % Theorem body font % (default is "\upshape")
{}                % Indent amount
{\bfseries}       % Theorem head font % (default is \mdseries)
{:}               % Punctuation after theorem head % default: no punctuation
{ }               % Space after theorem head
{}                % Theorem head spec
\theoremstyle{exercise}
\newtheorem{exercise}{Exercise}


\newtheoremstyle{example}
{}                % Space above
{}                % Space below
{\color{red}\slshape}        % Theorem body font % (default is "\upshape")
{}                % Indent amount
{\bfseries}       % Theorem head font % (default is \mdseries)
{.}               % Punctuation after theorem head % default: no punctuation
{ }               % Space after theorem head
{}                % Theorem head spec
\theoremstyle{example}
\newtheorem{example}{Example}

\newtheoremstyle{solution}
{}                % Space above
{}                % Space below
{\color{C5}}        % Theorem body font % (default is "\upshape")
{}                % Indent amount
{\bfseries }       % Theorem head font % (default is \mdseries)
{:}               % Punctuation after theorem head % default: no punctuation
{\newline}               % Space after theorem head
{}                % Theorem head spec
\theoremstyle{solution}
\newtheorem*{solution}{Solution}


\newcommand{\mcS}{\mathcal S}
\newcommand{\mcR}{\mathcal R}
\newcommand{\mcC}{\mathcal C}
\newcommand{\bS}{{\boldsymbol S}}
\newcommand{\bR}{{\boldsymbol R}}
\newcommand{\bC}{{\boldsymbol C}}
\newcommand{\ba}{{\boldsymbol a}}
\newcommand{\bb}{{\boldsymbol b}}
\newcommand{\bs}{{\boldsymbol s}}
\newcommand{\bff}{{\boldsymbol f}}
\newcommand{\br}{{\boldsymbol r}}
\newcommand{\bx}{{\boldsymbol x}}
\newcommand{\bt}{{\boldsymbol t}}
\newcommand{\bv}{{\boldsymbol v}}
\newcommand{\bu}{{\boldsymbol u}}
\newcommand{\bw}{{\boldsymbol w}}
\newcommand{\bc}{{\boldsymbol c}}
\newcommand{\be}{{\boldsymbol e}}
\newcommand{\bq}{{\boldsymbol q}}
\newcommand{\bphi}{{\boldsymbol \phi}}
\newcommand{\brho}{{\boldsymbol \rho}}
\newcommand{\btau}{{\boldsymbol \tau}}
\newcommand{\reals}{\mathbb R}
\newcommand{\ints}{\mathbb N}
\newcommand{\E}{\mathbb E}
\newcommand{\Prob}{\mathbb P}


%%%%%%%%%%%%%%%




%--------------------------------------------------------------------------------------------------------------------------------

\title{Exercise set 5}


\begin{document}

\maketitle

%%%%%%%%%%%%%%%%%%%%%%%%%%%%%%%%%%%%%%%%%%%%%%%%%%%%%%%%%%%%%%
%%%%%%%%%%%%%%%%%%%%%%%%%%%%%%%%%%%%%%%%%%%%%%%%%%%%%%%%%%%%%%
%  EXERCISES  
%%%%%%%%%%%%%%%%%%%%%%%%%%%%%%%%%%%%%%%%%%%%%%%%%%%%%%%%%%%%%%
%%%%%%%%%%%%%%%%%%%%%%%%%%%%%%%%%%%%%%%%%%%%%%%%%%%%%%%%%%%%%%

%---------------------------------------------------------------------------------
\begin{exercise}[A binary and normal predictor]\label{ex:binnorm}
Consider the a linear regression model 
\begin{equation*}
Y|(X_1,X_2) \sim {\rm Normal}(\beta_0 + \beta_1X_1 + \beta_2X_2,\sigma^2)
\end{equation*}
 where the two predictors obey
\begin{align*}
X_1 &\sim {\rm Bernoulli}(q)\\
X_2|X_1 &\sim {\rm Normal}(bX_1,\sigma_{2,1}^2)
\end{align*}
You can assume $\beta_0 =0$ for this problem. 
\begin{enumerate}[label=(\alph*)]
\item Derive a formulas for ${\rm cov}(X_1,X_2)$ and ${\rm var}(X_2)$ in terms of the model parameters. 
%Then use this to derive a formula for the correlation coefficient between $X_1$ and $X_2$, which we will denote as $\rho_{1,2}$. 
%\item Let 
%\begin{equation*}
%r^2 = 1- \frac{{\rm var}(Y|X)}{{\rm var}(Y)}.
%\end{equation*}
%Recall that this is the quantity which is being approximated by the $R^2$ reported in \verb!statsmodels!. 
%Derive a formula for $r^2$ in terms of the model parameters. 
\item Derive a formula for ${\rm cov}(Y,X_1)$ in terms of $\beta_1$, $q$, $\beta_2$ and $b$.  
%\begin{equation*}
%{\rm cov}(Y,X_1) =  (\beta_1q + \beta_2bq)(1-q)
%\end{equation*}
\item Explain how the formula you derived in part (b) is related to the equation for ${\rm cov}(Y,X_1)$ in the single predictor regression model (page 4 on week 3 notes). In particular, for what parameter values do the two formulas coincide? Your conclusion will be a particular case of what we saw to be true more generally (see week 5 notes) concerning the relationship between $\beta_1$ and the covariances in a regression model with two predictions. 
%\item ({\bf optional}) Derive the formula
%\begin{equation*}
%{\rm var}(Y) = q(1-q)\left(\beta_1^2 + \beta_2^2b^2 + 2 b \right) + \beta_2^2\sigma_{2|1}^2
%+ \sigma^2
%\end{equation*}
%You will need to use the formula from the additional exercise set: For two (not necessarily independent) random variables $Z$ and $Z'$
%\begin{equation*}
%{\rm var}(Z+Z') = {\rm var}(Z) + {\rm var}(Z') + 2{\rm cov}(Z,Z'). 
%\end{equation*}
%This is the formula in the ``addition and multiplication section'' on the \href{https://en.wikipedia.org/wiki/Variance#Properties}{wikipedia  page}.
\item ({\bf optional}) The calculations in part (c) allows us to solve an exercise in Chapter 8 in Demidenko's textbook \cite{demidenko2019advanced}, albeit in the more restrictive  context of a binary and normal predictor: 
Is it possible that $\beta_1$ and $\beta_2$ are {\bf both negative}, yet the (marginal) slope of $Y$ vs. $X_1$ is {\bf positive}? If so, generate simulated data where this is the case. 
\end{enumerate}
\end{exercise}





\begin{exercise}[Earnings data revisited] Consider the earnings data. This can be loaded with 
\begin{Verbatim}
df = pd.read_csv("https://raw.githubusercontent.com/avehtari
/ROS-Examples/master/Earnings/data/earnings.csv")
\end{Verbatim}
As in the previous exercise set, you will study the association between earnings and gender, but now using regression with multiple predictors. 

\begin{enumerate}[label=(\alph*)]
\item Perform a linear regression using \verb!statsmodels! with gender and height as predictors. 
\item Provide interpretations for each regression coefficient (like we did in class for the test score example). 
\item Which factor, height or gender is more important based on your analysis? 
\item Based one the fitted model, predict the chance that someone who is not male and is 5.8ft earns more than a male who is the same height? To get a sense for the importance (or lack-thereof) of the height predictor, compare this to the chance that a male earns more than a non-male (regardless of height).
\end{enumerate}
\end{exercise}

\begin{exercise}[Sample distribution]\label{ex:sampledist}
In the \href{https://colab.research.google.com/drive/1oIRgP_7-c5DGV1D2iz5nj406mZfJxUIG?usp=sharing}{notebook from class}, we wrote code to generate samples from the sample distribution of $(\hat{\beta}_1,\hat{\beta}_2)$ in the model 
\begin{align*}
X_1 &\sim {\rm Normal}(0,1). \\
X_2|X_1 &\sim {\rm Normal}(bX_1,1-b^2)\\
Y|(X_1,X_2) &\sim {\rm Normal}( \beta_1X_1 + \beta_2X_2,\sigma^2) 
\end{align*} 
Specifically, we had a function which takes $\beta_1$, $\beta_2$ and $\beta_0$ as inputs and returns a dataframe where the columns are the samples of $\hat{\beta}_1$ and $\hat{\beta}_2$ respectively. When we plotted the correlation coefficient as a function of $b$ values and estimates the correlation coefficient between $\hat{\beta}_1$ and $\hat{\beta}_2$, it was a decreasing line. 
%Consider the model from Exercise \ref{ex:binnorm}. We will study the sample distribution of the fitted regression coefficients $\hat{\beta}_1$ and $\hat{\beta}_2$. 
%\begin{enumerate}[label=(\alph*)]
%\item Let $q=0.3,b=2,\sigma_{2|1} = 0.3,\beta_1 = 10,\beta_2 = 2,\sigma=0.2$ and $n = 40$ (there are 40 data points). Generate $1000$ simulated data sets with these parameters values, and for each one save the fitted regression coefficients $\hat{\beta}_1$ and $\hat{\beta}_2$. Then make a scatter plot of $\hat{\beta}_2$ vs. $\hat{\beta}_1$. Before doing so, I recommend thinking for a moment about what you expect this to look like. What will the association between these two estimates be? Will it be negative or positive? 
\begin{enumerate}[label=(\alph*)]
\item What would happen if instead of plotting the correlation coefficient, we plotted ${\rm se}(\hat{\beta}_1)$ as a function of $b$? Would it increase? decrease? neither? Note that both $X_1$ and $X_2$ are standardized, so the distribution of $X_1$ values is not changed when we adjust $b$. In answering this question, you can either give a geometric intuition, or do a calculation. You should check your answer with simulations, but you still need to provide a detailed explanation. 
\item Is it possible to have large standard errors on all the $\hat{\beta}_i$ values (measured relative to the true values of course), but still have a large (meaning close to one) value of $R^2$? If so, for what parameter values does this happen? Run simulation(s) to support your answer. 
\end{enumerate}

\end{exercise}

%---------------------------------------------------------------------------------
%\begin{exercise}[Height vs. weight paradox]
%The following data has data concerning body weight as a function of exercises intensity. You can check that if we perform a regression using exercise intensity are our predictor and body weight as our response variable ($Y$), the data suggests that exercise increases body weight, counter to most of our intuition. Using a regression analysis with multiple predictors, try to reconcile this. 
%\begin{Verbatim}
%df = pd.read_csv("https://raw.githubusercontent.com
%/eugenedemidenko/advancedstatistics/master/RcodeData/simpson.csv")
%y = df.BodyW.values
%x2 = df.Sex.values
%x1 = df.ExInt.values
%
%fig,ax = plt.subplots(figsize=(5,2))
%ax.plot(x1,y,"o")
%ax.set_xlabel("exercise")
%ax.set_ylabel("body weight")
%\end{Verbatim}
%\end{exercise}


%---------------------------------------------------------------------------------


%---------------------------------------------------------------------------------
%\begin{exercise}[Earnings data]
%In class we fit the model 
%\begin{align*}
%y& = a_0 + a_{\rm male}x_{\rm male} + a_{\rm age}x_{\rm age} + a_{\rm black} x_{\rm black} \\
%&+ a_{\rm white}x_{\rm white} + a_{\rm hispanic} x_{\rm hispanic} + a_{\rm other}x_{\rm other} + \epsilon
%\end{align*}
%where $y$ is earnings to the data set:
%\begin{Verbatim}
%data = pd.read_csv("https://raw.githubusercontent.com
%/avehtari/ROS-Examples/master/Earnings/data/earnings.csv")
%data = data.dropna() 
%\end{Verbatim}
%
%Add height as a predictor to the model described above and perform a regression analysis. Discuss the conclusions of this analysis. In your discussion you should reference:
%\begin{itemize}
%\item Interpretation of regression coefficients
%\item The $p$ values for each regression coefficient
%\item The $R^2$ value
%\end{itemize}
%\end{exercise}


%



 \bibliographystyle{unsrt}
\bibliography{./../refs.bib}



\end{document}

\begin{exercise}[Collinearity]
Recall the example from class in which we generated samples from the sample distribution of $(\hat{\beta}_1,\hat{\beta}_2)$ given the model
\begin{align*}
Y &= \beta_1X_1 + \beta_2X_2 + \epsilon \\
X_2 &= cX_1 + \epsilon_x \\
X_1 &\sim {\rm Normal}(0,1). 
\end{align*} 
Specifically, we had a function which takes $\beta_1$, $\beta_2$ and $\beta_0$ as inputs and returns a dataframe where the columns are the samples of $\hat{\beta}_1$ and $\hat{\beta}_2$ respectively. When we plotted the correlation coefficient as a function of $c$ values and estimates the correlation coefficient between $\hat{\beta}_1$ and $\hat{\beta}_2$ it was a decreasing line. 

\begin{enumerate}
\item In the example above, what would happen if the standard deviation of $\hat{\beta}_1$ (which is essentially the standard error, or at least what the standard error is an approximation of) was plotted as a function of $c$? Note that both $x_1$ and $x_2$ are standardized, so the distribution of $x_1$ values is not changed when we adjust $c$. In answering this question, you can either give a geometric intuition, or do a calculation. You should check your answer with simulations, but you still need to provide a detailed explanation. 
\item What do the standard errors of $\hat{a}_i$ reported by statsmodels tell us about the ability of our model to predict new $y$ values based on existing $x$ values in a regression with multiple predictors? What about $R^2$? Is it possible to have large standard errors on all the $\hat{\beta}_i$ values, but still have a large $R^2$? Run simulation(s) to support your answer. 
\end{enumerate}
\end{exercise}