\documentclass[serif,mathserif]{beamer}
\usepackage{tikz}
\usepackage{pgfplots}
\usepackage{mathrsfs}
\usepackage{caption}
\usepackage{subfigure}
\usetikzlibrary{datavisualization}
\usetikzlibrary{arrows}
\usepackage{ragged2e}
\usepackage{svg}
\usepackage{mathtools}

  \newcommand{\mcS}{\mathcal S}
\newcommand{\mcR}{\mathcal R}
\newcommand{\mcC}{\mathcal C}
\newcommand{\mcL}{\mathcal L}
\newcommand{\mcA}{\mathcal A}
\newcommand{\bS}{{\boldsymbol S}}
\newcommand{\bR}{{\boldsymbol R}}
\newcommand{\bC}{{\boldsymbol C}}
\newcommand{\bX}{{\boldsymbol X}}
\newcommand{\bF}{{\boldsymbol F}}
\newcommand{\bG}{{\boldsymbol G}}
\newcommand{\bW}{{\boldsymbol W}}
\newcommand{\ba}{{\boldsymbol a}}
\newcommand{\bs}{{\boldsymbol s}}
\newcommand{\bff}{{\boldsymbol f}}
\newcommand{\br}{{\boldsymbol r}}
\newcommand{\by}{{\boldsymbol y}}
\newcommand{\bx}{{\boldsymbol x}}
\newcommand{\bv}{{\boldsymbol v}}
\newcommand{\bu}{{\boldsymbol u}}
\newcommand{\bc}{{\boldsymbol c}}
\newcommand{\bn}{{\boldsymbol n}}
\newcommand{\be}{{\boldsymbol e}}
\newcommand{\bq}{{\boldsymbol q}}
\newcommand{\balpha}{{\boldsymbol \alpha}}
\newcommand{\reals}{\mathbb R}
\newcommand{\ints}{\mathbb N}
\newcommand{\bbM}{\mathbb M}
\newcommand{\bbX}{\mathbb X}
\newcommand{\bbP}{\mathbb P}
\newcommand{\E}{\mathbb E}



\usepackage{amsmath}
%\usepackage{algorithm}



\newtheorem{thm}{Theorem}[section]
\newtheorem{defn}{Definition}[section]
\newtheorem{lem}{Lemma}[section]
\newtheorem{prop}{Proposition}[section]
\newtheorem{ass}{Assumption}
\newtheorem{cor}{Corollary}[section]

\usepackage{csvsimple}
\usepackage{./../beamerthemeETHAN-slides}




\begin{document}
%%%%%%%%%%%%%%%%%%%%%%%%%%%%%%%%%%%%%%%%%%%%%%%%%%%%%%%%%%%%

%%%%%%%%%%%%%%%%%%%%%%%%%%%%%%%%%%%%%%%%%%%%%%%%%%%%%%%%%%%
\begin{frame}
\centering
{\Large {Welcome to Math 50!}}
\centering



\end{frame}

%%%%%%%%%%%%%%%%%%%%%%%%%%%%%%%%%%%%%%%%%%%%%%%%%%%%%%%%%%%
\begin{frame}
\frametitle{What is the course about?}

\begin{itemize}
\item Math 50 is a course about {\bf extracting and interpreting relationships between variables} using real data. 
\item {\bf Linear and logistic regression models} are the mathematical frameworks for achieving this.  
\end{itemize}
\pause




\end{frame}

%%%%%%%%%%%%%%%%%%%%%%%%%%%%%%%%%%%%%%%%%%%%%%%%%%%%%%%%%%%
\begin{frame}
\frametitle{Comparison to other courses}



\end{frame}

%%%%%%%%%%%%%%%%%%%%%%%%%%%%%%%%%%%%%%%%%%%%%%%%%%%%%%%%%%%
\begin{frame} 
\begin{table}[h!]
\centering
\begin{tabular}{lll} 
& Topic & Overlap\\
\hline
\hline
Week 1 & Discrete probability models and simulations & 10, 20, 40, 60   \\
Week 2  & Sums and CLT  & 20, 40, 60 \\
Week 3  & Inference & 40, 70  \\
Week 4  & Regression single predictor & 10 \\
Week 5  & Regression with multiple predictors & 70  \\
Week 6  & Relationship to other models, Logistic regression & 70  \\
Week 7  & Overfitting and regularization & 70  \\
Week 8  & Bayesian inference &   \\
\end{tabular}
\caption{Weekly schedule (see course webpage for details) }
\label{tab:schedule}

\end{table}
\end{frame}


%%%%%%%%%%%%%%%%%%%%%%%%%%%%%%%%%%%%%%%%%%%%%%%%%%%%%%%%%%%
\begin{frame} 
\frametitle{Resources}

\end{frame}

%%%%%%%%%%%%%%%%%%%%%%%%%%%%%%%%%%%%%%%%%%%%%%%%%%%%%%%%%%%
\begin{frame}
\frametitle{Course structure}


\end{frame}

%%%%%%%%%%%%%%%%%%%%%%%%%%%%%%%%%%%%%%%%%%%%%%%%%%%%%%%%%%%
\begin{frame}
\frametitle{Policies}


\end{frame}

%%%%%%%%%%%%%%%%%%%%%%%%%%%%%%%%%%%%%%%%%%%%%%%%%%%%%%%%%%%
\begin{frame} 
\frametitle{How to succeed in this course}
\begin{itemize}
\item 
\end{itemize}

\end{frame}

%%%%%%%%%%%%%%%%%%%%%%%%%%%%%%%%%%%%%%%%%%%%%%%%%%%%%%%%%%%
\begin{frame} 
\frametitle{What to expect from me}

\begin{itemize}
\item {\bf Availability:} Office hours and xhours (4 hours total outside of class)
\item {\bf Responsiveness:} 24 hours on weekdays. If you'd like to request an extension, plan ahead
\item {\bf Grading:} No detailed grading of problem sets (graders will review for completeness). Detailed grading of midterm within 1 week. 
\item {\bf Material:} Course notes should be understood as a summary of material covered. Refer to textbooks/readings for technical details. 
\end{itemize}

\end{frame}





\end{document}

